
\section{Drøfting av potensielle problemer man kan møte på i det nye systemet/ hva som kan være vanskelig}


En av de største svakhetene ved vårt forslag er at vi velger å ikke endre på Its learning. Grunnen til dette er at Its learning er en pakkeløsning som det er veldig vanskelig og dyrt å gjøre endringer på. Systemet i seg selv har mye bra funksjonalitet, men problemet ligger i at forelesere ikke bruker systemet. Dette har vi tenkt å adressere ved å synliggjøre Its learning mer og ved å ha kursing i Its learning for forelesere. Dette håper vi vil gjøre at flere forelesere tar i bruk systemet.

En annen svakhet er at vi har valgt å bruke BuyPass til signering av klager på studentweb og dette innebærer at studentene må skaffe seg BuyPass. En stor risiko her er at studenter ikke skaffer seg BuyPass og dermed ikke tar i bruk løsningen, men siden det fortsatt vil være mulig å klage med papir vil BuyPass ikke legge til noen begrensinger til prosessen. BuyPass i seg selv er en sikker og stabil tjeneste som vil bidra til sikkerheten på studentweb. Brukere som velger å ikke bruke BuyPass skal kunne å skrive ut et klage-/begrunnelsesskjema inne på studentweb, som automatisk kan fylles ut med all relevant data som emnekode, fakultet osv.. Slik vil også brukere som ikke har BuyPass få nytte av systemet. 

Løsningen vår innebærer at man får samlet både eksamensweb og studentweb på samme sted. Dette er en stor fordel for brukere, men svakheten med denne løsningen er at man får et “single point of failure”, dvs. hvis studentweb går ned, er det heller ikke mulig å klage på eksamensresultater. Klaging på eksamensresultater har en tidsfrist og derfor er det viktig at det nye studenweb er stabil, robust og er lett å vedlikeholde.

En annen risiko er at det blir veldig mye informasjon inne på en side, derfor det vil være hensiktsmessig å se på brukergrensesnittet til studentweb og prøve å gjøre det mer oversiktlig og brukervennlig, slik at det blir lett å finne fram til den nye funksjonaliteten.
 	 	 	
Vi har ikke tenkt å gjøre noen endringer i innloggingsprossessen til studentweb. Grunnen til dette er at informasjonen som ligger inn på studentweb ikke regnes som sensitiv og derfor er det ikke noe behov for en høyere sikkerhet. Det skal imidlertid fortsatt være slik at sesjonen avsluttes etter 10 minutter hvis det ikke skjer noe aktivitet inne på siden. Dette synes vi er et bra tiltak for å hindre at en bruker glemmer å logge seg ut av systemet.

