\section{Konklusjon}

Denne rapporten har omtalt vårt forslag til nytt system, prosessen om hvordan vi kom frem til forslaget og begrensningene som følger med de valg som har blitt tatt. Forslagene
har rot i brukerundersøkelser med systemeiere og brukere i roller som faglærere, studenter og teknisk support. \\ 

Endringene vi har valgt er ikke en ny start på den perfekte integrerte
undervisningssituasjon, men en adapsjon gjort på med tanke på begrensninger og fordeler i eksisterende systemer og respekt for ressurser tilgjengelig for implementasjonsteamet. \\
Vår implementasjon vil ta en eksisterende løsning som er tilgjengelig for ett lite subsett av studenter og ansatte og bygge det inn i ett større system, brukt av alle studenter
og faglærere. Vi kommer et hakk nærmere papirløs hverdag og effektiviserer klageprosessen ved digital signering. \\

Vi har valgt å legge vekt på at man ikke nødvendigvis er nødt til å endre ett system for å få bedre utbytte av det, vi har tilgang til StudentWeb og kan gjøre de endringer vi vil.
Men i tilfelle med itslearning der forholdende ikke ligger til rette for endring føler vi det er viktigere å legge til rette for at faglærere er informert om de mulighetene de kan velge å utnytte i portalen både med tanke på papirløse prøver, skreddersydde øvingsopplegg og mulighetene for å dele og integrere informasjon på en hensiktsmessig måte.


